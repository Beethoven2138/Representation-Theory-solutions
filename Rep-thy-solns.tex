\documentclass{article}
\usepackage{graphicx} % Required for inserting images
\usepackage[english]{babel}
\usepackage{amssymb}
\usepackage{amsthm}
\usepackage{enumitem} 
\usepackage{amsmath}
\usepackage{amsfonts}
\usepackage{tikz}
\usetikzlibrary{matrix}
\usepackage[english]{babel}
\usepackage{mathtools}
\usepackage[a4paper, total={6in, 8in}]{geometry}
\usepackage{cite}
\title{Representation Theory Solutions}
\author{Saxon Supple}
\date{November 2024}
\usepackage{centernot}

\newtheorem{theorem}{Theorem}[section]
\newtheorem{definition}[theorem]{Definition}
\newtheorem{lemma}[theorem]{Lemma}
\newtheorem{proposition}[theorem]{Proposition}
\newtheorem{corollary}[theorem]{Corollary}
\newtheorem{example}[theorem]{Example}
\newtheorem{remark}[theorem]{Remark}
\newtheorem{exercise}[theorem]{Exercise}

\begin{document}

\maketitle

\begin{exercise}
Consider a basis of $V$ to be a linear isomorphism $\beta : \mathbb{F}^n \to V$ (where $n = \dim V$). Associating to a matrix $A \in \mathrm{GL}(n, \mathbb{F})$ the linear operator $\theta_A : \mathbb{F}^n \to \mathbb{F}^n$, show that the map
\[
\Phi : \mathrm{GL}(n, \mathbb{F}) \to \mathrm{GL}(V) : A \mapsto \beta \theta_A \beta^{-1}
\]
is an isomorphism of groups.
\end{exercise}
\begin{proof}
$\Phi$ is well-defined because the composition of isomorphisms is an isomorphism so $\beta\theta_A\beta^{-1}\in GL(V)\forall A\in GL(n,\mathbb{F})$. $\Phi(AB)=\beta \theta_{AB}\beta^{-1}=\beta \theta_{A}\beta^{-1}\beta \theta_{B}\beta^{-1}=\Phi(A)\Phi(B)$ so $\Phi$ is a homomorphism. Now let $\Phi(A)=\text{Id}_V$. Then $\beta \theta_A\beta^{-1}=\text{Id}_V\implies \theta_A\beta^{-1}=\beta^{-1}\text{Id}_V\implies \theta_A=\beta^{-1}\beta=\text{Id}_{\mathbb{F}^n}\implies A=I$. Thus $\Phi$ is injective. Given a $\phi\in GL(V)$ let $A$ be the matrix representing $\phi$ with respect to $\beta$. Then $\Phi(A)=\phi$. Thus $\Phi$ is surjective so is an isomorphism of groups.
\end{proof}

\begin{exercise}
Let $\phi: G \to \mathrm{GL}(V)$ be a representation of $G$. Show that the map
\[
\alpha: G \times V \to V : (g, v) \mapsto g \cdot v = \phi(g)(v)
\]
is a linear $G$-action.
\end{exercise}
\begin{proof}
Let $g,h\in G,v\in V$. Then $(gh)\cdot v=\phi(gh)(v)=\phi(g)(\phi(h)(v))=\phi(g)(h\cdot v)=g\cdot(h\cdot v)$.
$e\cdot v=\phi(e)(v)=\text{Id}_V(v)=v$. Thus $\alpha$ is a $G$-action.
Let $\lambda,\mu\in\mathbb{F}$ and $v,w\in V$. Then $g\cdot(\lambda v+\mu w)=\phi(g)(\lambda v+\mu w)=\lambda\phi(g)v+\mu\phi(g)w=\lambda(g\cdot v)+\mu(g\cdot w)$. Thus $\alpha$ is a linear $G$-action.
\end{proof}

\begin{exercise}
Let $\alpha : G \times X \to X : (g, x) \mapsto g \cdot x$ be a $G$-action and $\tilde{\alpha} : G \times \mathbb{F}X \to \mathbb{F}X : (g, f) \mapsto g \cdot f$ be its linearisation. Let $\{\delta_x : x \in X\}$ be the standard basis of $\mathbb{F}X$. 

\noindent Show that $g \cdot \delta_x = \delta_{g \cdot x}$. Deduce that the linearised action can be characterised by
\[
g \cdot \sum_{x \in X} \lambda_x \delta_x = \sum_{x \in X} \lambda_x \delta_{g \cdot x}.
\]
\end{exercise}
\begin{proof}
$(g\cdot\delta_x)(y)=\delta_x(g^{-1}\cdot y)$ which is $1$ iff $g^{-1}\cdot y=x\iff y=g\cdot x$ and $0$ otherwise. Thus $g \cdot \delta_x = \delta_{g \cdot x}$. By linearity, $g \cdot \sum_{x \in X} \lambda_x \delta_x=\sum_{x \in X} \lambda_x g\cdot\delta_x=\sum_{x \in X} \lambda_x \delta_{g \cdot x}$
\end{proof}


\begin{exercise}
Find $n$ different degree $1$ representations of $\mathbb{Z}_n$ over $\mathbb{C}$ and determine which are faithful.
\end{exercise}
\begin{proof}
$\rho_k:\mathbb{Z}_n\to\mathbb{C}^{*}:a\mapsto\omega^{ak}$ where $\omega=e^{\frac{2\pi i}{n}}$ and $0\leq k\leq n-1$ are $n$ degree $1$ representations. They're homomorphisms since $\rho_k(a+b)=\omega^{(a+b)k}=\omega^{ak}\omega^{bk}=\rho_k(a)\rho_k(b)$. $\rho_k$ is faithful when its image is the set of all $n$'th roots of unity which occurs when $\omega^k$ has order $n$ which occurs when $k$ is coprime to $n$.
\end{proof}

\begin{exercise}
Show that the only finite subgroups of the group $\mathbb{C}^*$ are the cyclic groups of $n$th roots of unity, for each positive integer $n$.
\end{exercise}
\begin{proof}
Let $H$ be a subgroup of $\mathbb{C}^*$ of order $n$ and let $h\in H$. Then $h^n=1$ by Lagrange's theorem so $h$ is an $n$th root of unity. Furthermore, since $H$ has order $n$ and only comprises $n$th roots of unity, $H$ must then contain all $n$th roots of unity and so be cyclic, generated by $e^{\frac{2\pi i}{n}}$.
\end{proof}

\begin{exercise}
The dihedral group $D_{2n}$ may be characterised as a group of order $2n$ generated by two elements $a$ and $b$, subject to the relations $a^n=b^2=1$ and $ba=a^{-1}b$.
\begin{enumerate}[label=(\roman*)]
    \item Show that any element of $D_{2n}$ can be written uniquely in the form $a^ib^j$ where $0\leq i<n$ and $0\leq j<2$.
    \item Now let $H$ be some other group with $h,k\in H$. What are necessary and sufficient conditions for there to exist a homomorphism $\theta:D_{2n}\to H$ with $\theta(a)=h$ and $\theta(b)=k$? Show that, then, such a homomorphism is unique.
    \item Show that $D_{2n}$ has a representation $\rho:D_{2n}\to GL(1,\mathbb{R})$ with $\rho(a)=1$ and $\rho(b)=-1$. What is the geometric significance of this representation?
    \item Write down a faithful matrix representation of $D_8$ of degree $2$.
\end{enumerate}
\end{exercise}
\begin{proof}
\begin{enumerate}[label=(\roman*)]
    \item Let $w$ be an element of $D_{2n}$ written as $x_1^{y_1}...x_n^{y_m}$ for $x_i\in\{a,b\},y_i\in\mathbb{Z}$ where $y_i=1$ if $x_i=b$. Suppose that there is some $x_i=b$ with $i\neq m$. Then $x_{i+1}=a$ so we can swap $b$ $y_{i+1}$ times with the rightwards $a$ so that $w=x_1^{y_1}...x_{i+1}^{-y_{i+1}}bx_{i+2}^{y_{i+2}}...x_{m}^{y_m}$. Again simplify if possible if $x_{i+2}=b$. Repeating this process will then give $w=a^{x}b^j$ where $0\leq j<2$. Then let $i=x+ln$ where $l\in\mathbb{Z}$ is such that $i$ is in the desired range. This shows that any element of $D_{2n}$ can be written in the desired form. To show uniqueness, note that $D_{2n}$ has order $2n$, each of the form $a^ib^j$ with $i$ and $j$ in the given ranges, and there are at most $2n$ possible elements given by $a^ib^j$ so we must have uniqueness in order to cover all $2n$ elements. 
    \item We must have $h^n=b^2=1$ and $kh=h^{-1}k$ as a necessary condition. $\theta$ is then given as $\theta(a^ib^j)=h^ik^j$. To show that $\theta$ is a homomorphism, let $v=a^i$ and $w=a^xb^y$. Then $\theta(vw)=\theta(a^ia^xb^y)=\theta(a^{i+x}b^y)=h^{i+x}k^y=h^ih^xk^y=\theta(v)\theta(w)$. Now let $v=a^ib$. Then $\theta(vw)=\theta(a^iba^xb^y)=\theta(a^{i-x}b^{y+1})=h^{i-x}k^{y+1}=h^ikh^xk^y=\theta(v)\theta(w)$. Thus the necessary conditions are also sufficient conditions for there to be a homomorphism. The homomorphism is also completely determined by its values in $a$ and $b$ so is unique.
    \item $a^n=(-1)^2=1$ and $(-1)1=1^{-1}(-1)$ so there exists such a representation. The representation encodes whether or not a reflection occurred on a regular $n$-gon.
    \item let $\rho(a)=\begin{pmatrix}
  \text{cos}(\frac{2\pi}{8}) & -\text{sin}(\frac{2\pi}{8})\\ 
  \text{sin}(\frac{2\pi}{8}) & \text{cos}(\frac{2\pi}{8})
\end{pmatrix}$ and let $\rho(b)=\begin{pmatrix}
  1 & 0\\ 
  0 & -1
\end{pmatrix}.$

$\rho(a^i)=\begin{pmatrix}
  \text{cos}(\frac{2\pi i}{8}) & -\text{sin}(\frac{2\pi i}{8})\\ 
  \text{sin}(\frac{2\pi i}{8}) & \text{cos}(\frac{2\pi i}{8})
\end{pmatrix}=I\iff i\in 8\mathbb{Z}\iff a^i=1$.

$\rho(a^ib)=\begin{pmatrix}
  \text{cos}(\frac{2\pi i}{8}) & \text{sin}(\frac{2\pi i}{8})\\ 
  \text{sin}(\frac{2\pi i}{8}) & -\text{cos}(\frac{2\pi i}{8})
\end{pmatrix}$ which is never $I$.

This $\rho$ has trivial kernel so is faithful.
\end{enumerate}
\end{proof}

\begin{exercise}
The quaternion group $Q_8$ is the subgroup of $GL(2,\mathbb{C})$ generated by the matrices\[A=\begin{pmatrix}
  i & 0\\ 
  0 & -i
\end{pmatrix},   B=\begin{pmatrix}
  0 & 1\\ 
  -1 & 0
\end{pmatrix}.\]
\begin{enumerate}[label=(\roman*)]
    \item Show that $A^4=1$, $A^2=B^2$ and $BAB^{-1}=A^{-1}$, and conclude that $Q_8$ is non-abelian.
    \item Show that any element of $Q_8$ can be written uniquely in the form $A^iB^j$ with $0\leq i<4$ and $0\leq j<2$. Thus confirm that $Q_8$ has order $8$.
    \item Is $Q_8$ isomorphic to $D_8$?
\end{enumerate}
\end{exercise}
\begin{proof}
\begin{enumerate}[label=(\roman*)]
    \item $BA=A^{-1}B\neq AB$.
    \item Let $W=X_1^{Y_1}...X_n^{Y_n}$ where $X_i$ is $A$ or $B$. Suppose that there is an $i$ such that $X_i=B$ and $X_{i+1}=A$. Then $W=X_1^{Y_1}...X_{i+1}^{-Y_{i+1}}X_i^{Y_i}...X_n^{Y_n}$. Repeat this until $W=A^pB^q$. Then let $r=p+4k$ such that $0\leq r<4$ and let $s=q+4k$ such that $0\leq s<4$. Then $W=A^rB^s$. If $s=0$ or $s=1$, let $i=r$ and $j=s$. If $s=2$, let $i=r+2\text{ mod }4$ and $j=0$. And if $s=3$, let $i=r+2\text{ mod }4$ and let $j=1$. Then $W=A^iB^j$ with $i$ and $j$ in the desired ranges.

    For uniqueness, it can be shown that $A$ and $B$ can generate $8$ distinct values, and each can be written as one of the $8$ possibilities of $A^iB^j$ so there can be no repetition so we have uniqueness.
    \item No. the elements of $Q_8$ has orders $1,4,2,4,4,4,4,4$ whereas the elements of $D_8$ have orders $1,4,2,4,2,$etc. The groups then have different numbers of elements of order $2$ so can't be isomorphic.
\end{enumerate}
\end{proof}

\begin{exercise}
Show that two matrix representations of degree one, $\rho_1:G\to GL(1,\mathbb{F})$ and $\rho_2:G\to GL(1,\mathbb{F})$ are isomorphic if and only if $\rho_1=\rho_2$.
\end{exercise}
\begin{proof}
$(\impliedby)$ Trivial. Take $\text{Id}_{\mathbb{F}}$ as the $G$-linear map.

$(\implies)$ Let $\theta:\mathbb{F}\to\mathbb{F}:v\mapsto\lambda v$ for $\lambda\in\mathbb{F}^*$ be a $G$-linear isomorphism. Then $\lambda\rho_1(g)=\theta\rho_1(g)=\rho_2(g)\theta=\lambda\rho_2(g)\implies\rho_1(g)=\rho_2(g)\forall g\in G$ so $\rho_1=\rho_2$.
\end{proof}

\begin{exercise}
Consider the real matrix representation of \( G = \mathbb{Z}_3 \) given by
\[
\rho : \mathbb{Z}_3 \to \mathrm{GL}(2, \mathbb{R}) : k \mapsto
\begin{pmatrix}
\cos \frac{2\pi k}{3} & -\sin \frac{2\pi k}{3} \\
\sin \frac{2\pi k}{3} & \cos \frac{2\pi k}{3}
\end{pmatrix}.
\]

Show that the corresponding \( G \)-module \( V = \mathbb{R}^2 \) is irreducible.

On the other hand, if we use the same matrices to define a complex representation
\[
\rho_{\mathbb{C}} : \mathbb{Z}_3 \to \mathrm{GL}(2, \mathbb{C}),
\]
show that the corresponding \( G \)-module \( V_{\mathbb{C}} = \mathbb{C}^2 \) is not irreducible.

\end{exercise}

\begin{proof}
Let $W\neq \{0\}$ be a $G$-submodule of $V$. Let $0\neq x\in W$. Then $\rho(1)(x)$ rotates $x$ by $\frac{2\pi}{3}$ radians and so $x$ and $\rho(1)(x)$ are linearly independent. Thus $\text{dim}(W)=2$ so $W=V$. Thus $V$ is irreducible.

Let $W=\{(\lambda,\lambda i),\lambda\in\mathbb{C}\}$. Given $\lambda\in\mathbb{C}^*$, $\rho(1)(\lambda,\lambda i)=\lambda(\text{cos}(\frac{2\pi}{3})-i\text{sin}(\frac{2\pi}{3}),\text{sin}(\frac{2\pi}{3})+i\text{cos}(\frac{2\pi}{3}))=\lambda(\text{cos}(\frac{2\pi}{3})-i\text{sin}(\frac{2\pi}{3}),(\text{cos}(\frac{2\pi}{3})-i\text{sin}(\frac{2\pi}{3}))i)=(\text{cos}(\frac{2\pi}{3})-i\text{sin}(\frac{2\pi}{3}))(\lambda,\lambda i)\in W$. Similarly, $\rho(2)(\lambda,\lambda i)\in W$ so $W$ is a $G$-submodule. $\text{dim}(W)=1<\text{dim}(\mathbb{C}^2)=2$ so $W\neq V_\mathbb{C}$. Thus $V_\mathbb{C}$ is not irreducible.
\end{proof}

\begin{exercise}
If \( G \) acts on a non-empty set \( X \), then the linearisation \( V = \mathbb{F}X \) contains two \( G \)-submodules:
\[
W_0 = \{ f \in \mathbb{F}X : \sum_{x \in X} f(x) = 0 \}, \quad
W_1 = \{ f \in \mathbb{F}X : f \text{ is constant} \}.
\]

Show that, if \( \mathrm{char} \, \mathbb{F} \) does not divide \( |X| \), then \( V \) is a direct sum \( V = W_0 \oplus W_1 \). What happens if \( \mathrm{char} \, \mathbb{F} \) does divide \( |X| \)?

\end{exercise}

\begin{proof}
Let $f\in W_0\cap W_1$. Then $f(x)=c\forall x\in X$ and $c|X|=0$. $|X|\neq0$ so $c=0$. Thus $W_0\cap W_1=\{0\}$. Let $f\in V$. Let $c=\sum_{x\in X}f(x)$. If $c=0$, then $f\in W_0$. Otherwise, define $g\in W_0$ by $g(x)=f(x)-\frac{c}{|X|}$ and $h\in W_1$ by $h(x)=\frac{c}{|X|}$. Then $f=g+h$. Thus $V=W_0\oplus W_1$.

If $\text{char }\mathbb{F}$ divides $|X|$ then $W_1\subseteq W_0$.
\end{proof}

\begin{exercise}
Let \( V \) and \( W \) be \( G \)-modules over a field \( \mathbb{F} \) and recall that 
\(\mathrm{Hom}_{\mathbb{F}}(V, W)\) denotes the vector space of all linear maps from \( V \) to \( W \).

\begin{itemize}
    \item[(a)] Show that \(\mathrm{Hom}_{\mathbb{F}}(V, W)\) becomes a \( G \)-module if we define \( g \cdot T \) by
    \[
    (g \cdot T)(v) = g \cdot \big(T(g^{-1} \cdot v)\big),
    \]
    for \( T \in \mathrm{Hom}_{\mathbb{F}}(V, W) \), \( g \in G \), and \( v \in V \).

    \item[(b)] The dual of \( V \) is \( V^* = \mathrm{Hom}_{\mathbb{F}}(V, \mathbb{F}) \). Show that \( V^* \) becomes a \( G \)-module if we define \( g \cdot f \) by
    \[
    (g \cdot f)(v) = f(g^{-1} \cdot v),
    \]
    for \( f \in V^* \), \( g \in G \), and \( v \in V \).
\end{itemize}

\end{exercise}
\begin{proof}
\begin{itemize}
    \item[(a)]Let $g,h\in G$. Then $((gh)\cdot T)(v)=(gh)\cdot(T((gh)^{-1}\cdot v))=g\cdot h\cdot(T(h^{-1}\cdot g^{-1}\cdot v))=(g\cdot h\cdot T)(v)$.

    Clearly $e\cdot T=T$.

    $(g\cdot(\lambda T + \mu F))(v)=g\cdot((\lambda T+\mu F)(g^{-1}\cdot v))=g\cdot (\lambda T(g^{-1}\cdot v)+\mu F(g^{-1}\cdot v))=\lambda(g\cdot T(g^{-1}\cdot v))+\mu(g\cdot F(g^{-1}\cdot v))=\lambda(g\cdot T)(v)+\mu(g\cdot F(v))$.
    \item[(b)]Consider $\mathbb{F}$ to be a $G$-module equipped with the trivial action. The result then follows from part $(a)$.
\end{itemize}
\end{proof}

\begin{exercise}
     Let \( V \) be a \( G \)-module and define \( V^G = \{ v \in V : g \cdot v = v \ \forall g \in G \} \).
    \begin{enumerate}
        \item[(a)] Show that \( V^G \) is a \( G \)-submodule of \( V \).
        \item[(b)] If \( U \) and \( V \) are \( G \)-modules and \( \mathrm{Hom}_F(U, V) \) is made into a \( G \)-module as in Exercise 2.4(a), show that \( \mathrm{Hom}_F(U, V)^G = \mathrm{Hom}_G(U, V) \), the subspace of \( G \)-linear maps.
    \end{enumerate}
\end{exercise}
\begin{proof}
\begin{enumerate}
    \item[(a)] Let $v,w\in V^G$ and let $\lambda,\mu\in\mathbb{F}$. Then given any $g\in G$, $g\cdot(\lambda v+\mu w)=\lambda(g\cdot v)+\mu(g\cdot w)=\lambda v+\mu w$. Thus $V^G$ is a linear subspace of $V$. Furthermore, $h\cdot(g\cdot v)=h\cdot v=v\forall h\in G$ so $g\cdot v\in V^G$. Thus $V^G$ is closed under the group action so is a $G$-submodule of $V$.
    \item[(b)] Let $\theta\in\text{Hom}_{\mathbb{F}}(U,V)^G$. Let $u\in U,g\in G$. Then $g^{-1}\cdot\theta=\theta$ so $\theta(u)=g^{-1}\cdot(\theta((g^{-1})^{-1}\cdot u))\implies g\cdot\theta(u)=\theta(g\cdot u)$. Thus $\text{Hom}_{\mathbb{F}}(U,V)^G\subseteq \text{Hom}_G(U,V)$. Now let $\theta\in\text{Hom}_G(U,V)$. Then $\theta(g^{-1}\cdot u)=g^{-1}\cdot\theta(u)\implies (g\cdot\theta)(u)=\theta(u)$. Thus $\text{Hom}_{\mathbb{F}}(U,V)^G= \text{Hom}_G(U,V)$.
\end{enumerate}
\end{proof}


\begin{exercise}
     Let \( V = U \oplus W \) be a direct sum of \( G \)-modules. Show that for any \( G \)-module \( X \)
    \[
    \mathrm{Hom}_G(V, X) \cong \mathrm{Hom}_G(U, X) \oplus \mathrm{Hom}_G(W, X)
    \]
    and thus
    \[
    \dim \mathrm{Hom}_G(V, X) = \dim \mathrm{Hom}_G(U, X) + \dim \mathrm{Hom}_G(W, X).
    \]
\end{exercise}
\begin{proof}
We have a $G$-linear projection $\pi:V\to V:u+w\mapsto w$. Now consider $\Psi:\text{Hom}_G(V,X)\to \text{Hom}_G(V,X):\theta\mapsto\theta\circ\pi$. Given $\lambda,\mu\in\mathbb{F},\theta,\phi\in\text{Hom}_G(V,X)$ we have $\Psi(\lambda\theta+\mu\phi)=(\lambda\theta+\mu\phi)\circ\pi=\lambda\theta\circ\pi+\mu\phi\circ\pi=\lambda\Psi(\theta)+\mu\Psi(\phi)$ so $\Psi$ is linear. Furthermore, $\Psi^2(\theta)=\theta\circ\pi^2=\theta\circ\pi=\Psi(\theta)$ so $\Psi$ is a projection. $\theta\in\text{Ker }\Psi\iff\theta\circ\pi=0\iff\theta_{|W}=0$ so $\text{Ker }\Psi\cong\text{Hom}_G(U,X)$. Also, $\theta\in\text{Im }\Psi\iff\exists\phi\in\text{Hom}_G(V,W):\theta=\phi\circ\pi\iff \theta_{|U}=0$ so $\text{Im }\Psi\cong\text{Hom}_G(W,X)$. $\text{Hom}_G(V,X)=\text{Ker }\Psi\oplus\text{Im }\Psi$ so $\text{Hom}_G(V,X)\cong\text{Hom}_G(U,X)\oplus\text{Hom}_G(W,X)$.
\end{proof}

\begin{exercise}
     Let \( U \) and \( X \) be two irreducible \( G \)-modules over \( \mathbb{C} \).
    \begin{enumerate}
        \item[(a)] Use Schur's Lemma to show that
        \[
        \dim \mathrm{Hom}_G(U, X) =
        \begin{cases}
        1 & \text{if } U \cong X, \\
        0 & \text{if } U \ncong X.
        \end{cases}
        \]
        \item[(b)] Deduce that, if \( X \) is an irreducible \( G \)-module and \( V = V_1 \oplus \cdots \oplus V_n \) is a direct sum of irreducible \( G \)-modules \( V_1, \dots, V_n \) with \( V_i \cong X \) for \( 1 \leq i \leq k \) and \( V_j \ncong X \) for \( j > k \), then
        \[
        \dim \mathrm{Hom}_G(V, X) = k.
        \]
    \end{enumerate}
\end{exercise}

\begin{proof}
\begin{enumerate}
    \item[(a)] Let $U\not\cong X$. Then by Schur's lemma, the only $G$-linear map in $\text{Hom}_G(U,X)$ is $0$.

    Now let $U\cong X$. Then by Schur's lemma, every $\theta\in \text{Hom}_G(U,X)$ is either $0$ or an isomorphism. Let $\theta\in \text{Hom}_G(U,X)$ be an isomorphism and let $\phi\in \text{Hom}_G(U,X)$. Then $\theta^{-1}\circ\phi=\lambda\text{Id}_U$ so $\phi=\lambda\theta$. Thus $\text{Hom}_G(U,X)=\langle\theta\rangle$ so $\text{dim Hom}_G(U,X)=1$.
    \item[(b)]  $\text{dim Hom}_G(V,X)=\sum_{i=1}^n\text{dim Hom}_G(V_i,X)=\sum_{i=1}^k\text{dim Hom}_G(V_i,X)=k$.
\end{enumerate}
\end{proof}

\begin{exercise}
     Let \( G = S_3 \) act tautologically on \( X = \{1, 2, 3\} \). Consider the linearisation \( V = \mathbb{C}X \) and (as in Example 1.26(b) of the notes) the submodule
    \[
    W_0 = \left\{ f \in \mathbb{C}X : \sum_{x \in X} f(x) = 0 \right\}.
    \]
    Show that \( W_0 \) is equivalent to the (unique) irreducible representation of degree 2 found in Theorem 1.31 of the notes.
\end{exercise}
\begin{proof}
We have that $\mathbb{C}X=W_0\oplus W_1$ where $W_1=\{f\in\mathbb{C}X:f\text{ is constant}\}$. $\text{Dim }W_1=1$ so $\text{Dim }W_0=\text{Dim }\mathbb{C}X-1=3-1=2$. Let $f\in W_0$ with $f\neq 0$ such that $\langle f\rangle$ is a $G$-submodule. WLOG say that $f(1)\neq 0$. Let $g=(12)\cdot f$ so that $g(1)=f(2),g(2)=f(1)$ and $g(3)=f(3)$. $g$ is a scalar multiple of $f$ so $g=\frac{f(2)}{f(1)}f$ and $g=f$ so $f(1)=f(2)$ which makes $f(3)=-2f(1)$. Now consider $h=(13)\cdot f$ so that $h(1)=f(3),h(2)=f(2)$ and $h(3)=f(1)$. Then $h=f$ and $h=\frac{f(3)}{f(1)}f=\frac{-2f(1)}{f(1)}f=-2f$ so $f=0$. But we assumed that $f\neq 0$ giving a contradiction. Thus $W_0$ is irreducible and so is equivalent to the unique irreducible representation of $S_3$ of degree $2$.
\end{proof}


\begin{exercise}
     Show that, over \( \mathbb{C} \), all irreducible representations of the dihedral group \( D_{2n} \) have degree 1 or 2 and classify the irreducible representations of \( D_{2n} \) up to equivalence.
    \begin{itemize}
        \item (Hint: follow the method of Section 1.9 of the notes, i.e., the case of \( D_6 \).)
    \end{itemize}
\end{exercise}

\begin{proof}
$D_{2n}$ is generated by $a,b$ subject to the relations $a^n=b^2=1$ and $ba=a^{-1}b$. Let $V$ be an irreducible $D_{2n}$ module, with corresponding representation $\rho:D_{2n}\to GL(V)$. The action of $D_{2n}$ is determined by the actions of $a$ and $b$. The operator $\rho(a):V\to V$ has an eigenvector $v$ with eigenvalue $\omega$ so $a\cdot v=\rho(a)(v)=\omega v$. Furthermore, $\rho(a)^n=\rho(a^n)=\rho(1)=\text{Id}_V$ so $\omega^n=1$. Thus $\omega=e^{\frac{2\pi ik}{n}}$ for some $k\in\{0,...,n-1\}$. Also, $a^{-1}\cdot v=\rho(a^{-1})(v)=\rho(a)^{-1}(v)=\omega^{-1}v$. Now consider $b\cdot v$. $a\cdot(b\cdot v)=(ab)\cdot v=(ba^{-1})\cdot v=b\cdot(\omega^{-1}v)=\omega^{-1}(b\cdot v)$. Also $b\cdot(b\cdot v)=1\cdot v=v$. Thus $\text{span}\{v,b\cdot v\}$ is a $G$-submodule. $V$ is irreducible so $V=\text{span}\{v,b\cdot v\}$. Thus an irreducible $D_{2n}$-module is at most two dimensional.

\textbf{Case 1.} $1\leq k\leq \lfloor\frac{n-1}{2}\rfloor$ so $\omega\neq\omega^{-1}$. Then $v$ and $b\cdot v$ are eigenvectors of $\rho(a)$ with different eigenvalues so are linearly independent. Using the basis $\{v,b\cdot v\}$ to give a linear isomorphism $\beta:\mathbb{C}^2\to V$, $V$ is equivalent to the matrix representation $D_{2n}\to GL(2,\mathbb{C})$ determined by\[a\mapsto
\begin{pmatrix}
\omega & 0 \\
0 & \omega^{-1}
\end{pmatrix},b\mapsto\begin{pmatrix}
0 & 1 \\
1 & 0
\end{pmatrix}.\]The only non-zero proper subspaces of $V$ which are closed under the action of $a$ are the eigenspaces $\text{span}\{v\}$ and $\text{span}\{b\cdot v\}$ and these are not closed under the action of $b$. Thus $V$ is irreducible.
Let $\rho_1,\rho_2$ be representations for when $k=k_1,k_2$ respectively.
The cases $k_1=i$ and $k_2=n-i$ are equivalent since changing $k$ swaps $\omega$ and $\omega^{-1}$ so an equivalence can be obtained by swapping $v$ and $b\cdot v$. Otherwise $(e^{\frac{2\pi ik_1}{n}},e^{\frac{-2\pi ik_1}{n}})\neq(e^{\frac{2\pi ik_2}{n}},e^{\frac{-2\pi ik_2}{n}})$ and $(e^{\frac{2\pi ik_1}{n}},e^{\frac{-2\pi ik_1}{n}})\neq(e^{\frac{-2\pi ik_2}{n}},e^{\frac{2\pi ik_2}{n}})$ so $\rho_1$ and $\rho_2$ cannot be equivalent, since equivalent linear maps have the same eigenvalues.

\textbf{Case 2.} $\omega=\omega^{-1}$. Then $\omega^2=1\implies\omega=\pm1$. $\text{span}\{v+b\cdot v\}$ and $\text{span}\{v-b\cdot v\}$ are closed under the actions of $a$ and $b$ so are $D_{2n}$-submodules. Since $V$ is irreducible we must have that one of them is trivial with $V$ equal to the other, since otherwise they would be subspaces of different eigenspaces and so be distinct submodules. First suppose that $v=-b\cdot v$. Then $\rho(b)=-1$.

Now suppose that $v=b\cdot v$. Then $\rho(b)=1$. $\rho(a)=1$ and $\rho(a)=-1$ both work, however $\rho(a)$ is only possible when $n$ is even.

None of these representations are equivalent since $\rho(a)$ and $\rho(b)$ have different pairs of eigenvalues for each degree one representation $\rho$.
\end{proof}


\begin{exercise}Consider the following representation of the additive group $\mathbb{Z}$:
\[
\rho : \mathbb{Z} \to \text{GL}(2, \mathbb{F}) : n \mapsto 
\begin{pmatrix}
1 & n \\
0 & 1
\end{pmatrix}.
\]
\begin{itemize}
    \item[(a)] Show that there is only one 1-dimensional submodule of $\mathbb{F}^2$, as a $\mathbb{Z}$-module (via $\rho$).
    \item[(b)] Deduce that Maschke's Theorem can fail for infinite groups and even for finite groups when $\mathbb{F}$ has characteristic $p > 0$.
\end{itemize}
\end{exercise}
\begin{proof}
\begin{itemize}
    \item[(a)] Let $\langle(a,b)\rangle$ be a $1$-dimensional submodule of $\mathbb{F}^2$. Then $\rho(n)(a,b)=(a+nb,b)=\lambda_n(a,b)$ for some $\lambda_n\in\mathbb{F}$. If $b\neq0$ then $\lambda_n=1$ so $a+nb=a\forall n\in\mathbb{Z}$ which is impossible. Thus we need $b=0$. $\langle(a_1,0)\rangle=\langle(a_2,0)\rangle$ for any non-zero $a_1,a_2\in\mathbb{F}$ so $\langle(1,0)\rangle$ is the only $1$-dimensional submodule of $\mathbb{F}^2$.
    \item[(b)] There would need to be another $1$-dimensional submodule of $\mathbb{F}^2$ for Maschke's theorem to hold. Suppose $\mathbb{F}$ has characteristic $p>0$ and $\mathbb{F}^2$ has representation \[
\rho_p : \mathbb{Z}_p \to \text{GL}(2, \mathbb{F}) : n \mapsto 
\begin{pmatrix}
1 & n \\
0 & 1
\end{pmatrix}.
\] Then $\rho(x)=\rho([x]_p)$ so as before there is only one $1$-dimensional submodule of $\mathbb{F}^2$.
\end{itemize}
\end{proof}

\textbf{[Assume now that the base field $\mathbb{F}$ is $\mathbb{C}$ and $G$ is a finite group.]}

\begin{exercise} Let $V$ be a nonzero $G$-module that is not irreducible. Show that there is a $G$-linear map $T : V \to V$ which is neither zero nor invertible. (\textit{Hint: use Maschke's Theorem.})
\end{exercise}

\begin{proof}
By Maschke's theorem, $V=U\oplus W$ for non-zero $G$-submodules $U$ and $W$. Let $\pi$ be the projection map onto $U$. $\pi$ is then a $G$-linear map, is non-zero (since its image is $U$ which is non-zero) and is not invertible (since its kernel is $W$ which is non-zero).
\end{proof}

\begin{exercise} Let $U$ be an irreducible $G$-module and $V$ any $G$-module.
\begin{itemize}
    \item[(a)] Show that $V$ has an irreducible submodule isomorphic to $U$ if and only if there is a non-zero $G$-linear map $U \to V$.
    \item[(b)] Show that $V$ has an irreducible submodule isomorphic to $U$ if and only if there is a non-zero $G$-linear map $V \to U$.
\end{itemize}
\end{exercise}
\begin{proof}
\begin{itemize}
    \item[(a)]$(\implies)$ Let $W$ be an irreducible $G$-submodule of $V$ isomorphic to $U$. let $\theta:U\to W$ be the isomorphism. Then $\phi:U\to V$ given by $\phi(u)=\theta(u)\forall u\in U$ is a non-zero $G$-linear map.

    $(\impliedby)$ Let $T:U\to V$ be a non-zero $G$-linear map. By the first isomorphism theorem, $U/\text{Ker }T\cong\text{Im }T$. $\text{Ker }T$ is a submodule of $U$ so can be either $\{0\}$ or $U$ since $U$ is irreducible. However, $T\neq 0$ so $\text{Ker }T=\{0\}$. Thus $\text{Im }T$ is a submodule of $V$ which is isomorphic to $U$. Furthermore, $\text{Im }T$ must be irreducible, since otherwise $U$ wouldn't be irreducible.
    \item[(b)] $(\implies)$ Let $W$ be the irreducible submodule of $V$ isomorphic to $U$. Let $\theta:W\to U$ be the isomorphism and let $\pi:V\to W$ be the projection map onto $W$ which exists by Maschke's theorem. Then $\theta\circ\pi:V\to U$ is a non-zero $G$-linear map.

    $(\impliedby)$ Let $T:V\to U$ be a non-zero $G$-linear map. $\text{Im }T$ is either $\{0\}$ or $U$ since $U$ is irreducible. However, if $\text{Im }T=\{0\}$ then $T=0$; a contradiction. Thus $\text{Im }T=U$. By Maschke's theorem there exists a $G$-submodule $W$ such that $V=\text{Ker }T\oplus W$. Let $\theta:W\to U$ be the restriction of $T$ to $W$. Then $\theta$ is injective with image $U$ so $U\cong W$. As before, $W$ must also be irreducible.
\end{itemize}
\end{proof}


\begin{exercise} Suppose that all irreducible $G$-modules are 1-dimensional. Show that $G$ is abelian. (\textit{Hint: take a faithful $G$-module and decompose it into a direct sum of irreducible $G$-modules.})
\end{exercise}
\begin{proof}
Consider the regular representation of $G$ with corresponding representation $\rho:G\to GL(\mathbb{C}G)$. By Maschke's theorem we can decompose $\mathbb{C}G$ into a direct sum of irreducible $G$-modules. Since all irreducible $G$-modules are $1$-dimensional, we have $\mathbb{C}G=V_1\oplus...\oplus V_{|G|}$ where each $V_i$ has degree $1$. Let $g\in G$ and let $v_i\in V_i\backslash\{0\}$ so that $v_1,...,v_{|G|}$ form a basis. Then $\rho(g)(v_i)=\lambda_i(g)v_i\forall i$ where $\lambda_i(g)\neq 0$ since $V_i$ is $1$-dimensional. The matrix representing $\rho(g)$ with respect to the basis $v_1,...,v_{|G|}$ is then diagonal for every $g\in G$ and diagonal matrices commute under multiplication so the image of $\rho$ is abelian. $\rho$ is faithful so by the first isomorphism theorem $G\cong G/\{0\}=G/\text{Ker }\rho\cong\text{Im }\rho$. Thus $G$ is abelian.
\end{proof}

\begin{exercise} Let $V$ be a 2-dimensional $G$-module and $\rho : G \to \text{GL}(V)$ the associated representation. Show that if $\text{Im } \rho$ is not abelian, then $V$ is irreducible. Deduce that the following degree 2 representation of $D_{2n}$ $(n \geq 3)$ is irreducible:
\[
\rho : D_{2n} \to \text{GL}(2, \mathbb{C}) : 
\tau \mapsto 
\begin{pmatrix}
\omega & 0 \\
0 & \omega^{-1}
\end{pmatrix}, \quad 
\sigma \mapsto 
\begin{pmatrix}
0 & 1 \\
1 & 0
\end{pmatrix},
\]
where $\omega \in \mathbb{C}$ is any $n$th root of unity such that $\omega \neq \omega^{-1}$. Here $\tau$ and $\sigma$ are generators of $D_{2n}$ subject to the relations $\sigma^2 = 1$, $\tau^n = 1$, and $\tau\sigma = \sigma\tau^{-1}$.
\end{exercise}
\begin{proof}
Suppose that $V$ is reducible. Then by Maschke's theorem $V=U\oplus W$ where $U$ and $W$ have dimension $1$. We have distinct $g,h\in G$ such that $\rho(g)\rho(h)\neq\rho(h)\rho(g)$. Since $U$ is $1$-dimensional, $\rho(g)_{|U}=\lambda\text{Id}_U,\rho(h)_{|U}=\mu\text{Id}_U$ for some non-zero $\lambda,\mu$. Similarly, $\rho(g)_{|W}=\alpha\text{Id}_W,\rho(h)_{|W}=\beta\text{Id}_W$. Let $u,w$ be non-zero elements of $U$ and $W$ respectively so that they form a basis. Then with respect to the basis, $\rho(g)$ is represented by \[\begin{pmatrix}
\lambda & 0 \\
0 & \alpha
\end{pmatrix}\] and $\rho(h)$ is represented by\[\begin{pmatrix}
\mu & 0 \\
0 & \beta
\end{pmatrix}.\] But then $\rho(g)\rho(h)$ and $\rho(h)\rho(g)$ are both represented by \[\begin{pmatrix}
\lambda\mu & 0 \\
0 & \alpha\beta
\end{pmatrix}.\] so are equal; a contradiction. Thus $V$ must be irreducible.


\[\begin{pmatrix}
\omega & 0 \\
0 & \omega^{-1}
\end{pmatrix}\begin{pmatrix}
0 & 1 \\
1 & 0
\end{pmatrix}=
\begin{pmatrix}
0 & \omega \\
\omega^{-1} & 0
\end{pmatrix}.\]
\[
\begin{pmatrix}
0 & 1 \\
1 & 0
\end{pmatrix}\begin{pmatrix}
\omega & 0 \\
0 & \omega^{-1}
\end{pmatrix}=
\begin{pmatrix}
0 & \omega^{-1} \\
\omega & 0
\end{pmatrix}.
\]
Thus $\text{Im }\rho$ is not abelian so the representation is irreducible.
\end{proof}


Below \( G \) is a finite group and the base field is \( \mathbb{C} \).  
Also, \( C_n \) denotes the group of \( n \)-th roots of unity, a cyclic subgroup of \( \mathbb{C}^* = \mathbb{C} \setminus \{ 0 \} \).

\begin{exercise}
     Prove directly that, if \( V \) is a \( G \)-module with irreducible decomposition 
    \[
    V = V_1 \oplus \cdots \oplus V_n
    \]
    and \( W \) is any irreducible submodule of \( V \), then \( W \cong V_j \) for some \( j \).
\end{exercise}
\begin{proof}
Let $\pi_i:W\to V_i$ be the projection map onto $V_i$ restricted to $W$. $\text{Id}_W=\sum_i\pi_i$ is non-zero so there must be a non-zero $\pi_i$. By Schur's lemma, this is then an isomorphism.
\end{proof}

\begin{exercise}
     Recall that the \textbf{centre} \( Z(G) \) of \( G \) is the subgroup of \( G \) defined by
    \[
    Z(G) = \{ z \in G : zg = gz \text{ for all } g \in G \}.
    \]
    Let \( V \) be an irreducible \( G \)-module with the corresponding representation \( \rho : G \to \mathrm{GL}(V) \).

    \begin{enumerate}
        \item[(a)] Show that, for each \( z \in Z(G) \), there is \( \lambda(z) \in \mathbb{C}^* \) such that, for all \( v \in V \),
        \[
        z\cdot v = \lambda(z)v.
        \]
        (Hint: use Schur's Lemma.)

        \item[(b)] Show that \( z \mapsto \lambda(z) \) is a group homomorphism \( Z(G) \to \mathbb{C}^* \).

        \item[(c)] Deduce that, if \( \rho \) is faithful, then \( Z(G) \) is cyclic. (Hint: consider \( \rho(Z(G)) \) as a subgroup of \( \mathbb{C}^* \).)

        \item[(d)] Which of the following groups have a faithful irreducible representation: \( C_n \), \( D_8 \), \( C_2 \times D_8 \)?
    \end{enumerate}
\end{exercise}
\begin{proof}
\begin{enumerate}
    \item[(a)] Given any $g\in G$, $z\cdot(g\cdot v)=(zg)\cdot v=(gz)\cdot v=g\cdot(z\cdot v)$ so $\rho(z)$ is $G$-linear. Thus by Schur's lemma $\rho(z)=\lambda(z)\text{Id}$ for some $\lambda(z)\in\mathbb{C}^*$ (since the map is an isomorhpsm) so $z\cdot v=\lambda(z)v$.
    \item[(b)] $Z(G)\to GL(V):z\mapsto\lambda(z)\text{Id}$ is a group homomorphism so $z\mapsto\lambda(z)$ clearly is as well.
    \item[(c)] By the first isomorphism theorem, $Z(G)\cong\rho(Z(G))$ which we can consider to be a subgroup of $\mathbb{C}^*$. All subgroups of $\mathbb{C}^*$ are cyclic so $Z(G)$ is cyclic.
    \item[(d)] $C_n$ has a faithful irreducible representation given by $\rho:C_n\to \mathbb{C}^*:\omega\mapsto\omega$. 

    $D_8$ has a faithful irreducible representation $\rho:D_8\to GL(n,\mathbb{C})$ given by\[\rho(\tau)=\begin{pmatrix}
\omega & 0 \\
0 & \omega^{-1}
\end{pmatrix},\rho(\sigma)=\begin{pmatrix}
0 & 1 \\
1 & 0
\end{pmatrix}\] where $\omega=e^\frac{2\pi i}{4}$.

$Z(D_8)=\{e,\tau^2\}$ so $Z(C_2\times D_8)=\{(1,e),(-1,e),(1,\tau^2),(-1,\tau^2)\}\cong \mathbb{Z}_2\times\mathbb{Z}_2$ which is not cyclic so there is no faithful irreducible representation of $C_2\times D_8$.
\end{enumerate}
\end{proof}

\begin{exercise}
     Decompose the regular module \( \mathbb{C}C_3 \) as a direct sum of irreducible submodules.
\end{exercise}
\begin{proof}
let $\omega=e^{\frac{2\pi i}{3}}$. $C_3$ is abelian so every irreducible submodule has dimension $1$. Furthermore, every representation $\rho:C_3\to\mathbb{C}C_3$ is equivalent to precisely one representation $\chi_i:C_3\to\mathbb{C}^*:g\mapsto g^i$ for $i\in\{0,1,2\}$. Thus if $\langle f\rangle$ is an irreducible submodule of $\mathbb{C}C_3$, then $\omega\cdot f=\omega^i f$. One irreducible submodule is the one generated by $f(g)=1\forall g\in C_3$ which is equivalent to $\chi_0$. Now consider $h(g)=g\forall g\in C_3$. Then $\omega^2\cdot h=\omega h$ and $\omega\cdot h=\omega^2 h$ so $\langle h\rangle$ is a submodule which is equivalent to $\chi_2$. Finally consider $p(g)=g^2\forall g\in C_3$. Then $\omega\cdot p=\omega p$ and $\omega^2\cdot p=\omega^2 p$ so $\langle p\rangle$ is an irreducible submodule equivalent to $\chi_i$. $\mathbb{C}C_3$ has dimension $3$ so $\mathbb{C}C_3=\langle f\rangle\oplus\langle h\rangle\oplus\langle p\rangle$.
\end{proof}

\begin{exercise}
     Let \( \tau = (123) \) and \( \sigma = (12) \), which generate \( S_3 \). Decompose the regular module \( \mathbb{C}S_3 \) into a direct sum of its irreducible submodules by completing the following steps.
    
    \begin{enumerate}
        \item[(a)] Show that the subspaces 
        \[
        U_1 = \text{span}\{\delta_e, \delta_{(123)}, \delta_{(132)}\}, \quad 
        U_2 = \text{span}\{\delta_{(12)}, \delta_{(23)}, \delta_{(13)}\}
        \]
        of \( \mathbb{C}S_3 \) are closed under the action of \( \tau \).

        \item[(b)] Find the eigenvectors \( v_1, v_2, v_3 \) of \( \tau \) in \( U_1 \) and compute \( \sigma v_1, \sigma v_2, \sigma v_3 \).

        \item[(c)] Find 4 irreducible submodules \( V_1, V_2, V_3, V_4 \) of \( \mathbb{C}S_3 \) of degrees 1, 1, 2, 2, respectively, so that 
        \[
        \mathbb{C}S_3 = V_1 \oplus V_2 \oplus V_3 \oplus V_4.
        \]
        (Hint: recall the classification of representations of \( S_3 \) in Section 1.7.)
    \end{enumerate}
\end{exercise}
\begin{proof}
\begin{enumerate}
    \item[(a)]$(123)\cdot\delta_e=\delta_{(123)},(123)\cdot\delta_{(123)}=\delta_{(132)},(123)\cdot\delta_{(132)}=\delta_{e}$ so $U_1$ is closed under $\tau$. $(123)\cdot\delta_{(12)}=\delta_{(13)},(123)\cdot\delta_{(23)}=\delta_{(21)},(123)\cdot\delta_{(13)}=\delta_{(23)}$ so $U_2$ is closed under $\tau$.
    \item[(b)]The action of $\tau$ with respect to $\delta_e,\delta_{(123)},\delta_{(132)}$ is represented by\[\begin{pmatrix}
0 & 0 & 1 \\
1 & 0 & 0 \\
0 & 1 & 0 \\
\end{pmatrix}\] which has eigenvectors \[\begin{pmatrix}
    1\\
    1\\
    1\\
\end{pmatrix},\begin{pmatrix}
    \omega\\
    \omega^2\\
    1\\
\end{pmatrix},\begin{pmatrix}
    \omega^2\\
    \omega\\
    1\\
\end{pmatrix}\] with corresponding eigenvalues\[\lambda_1=1,\lambda_2=\omega^2,\lambda_3=\omega.\] where $\omega=e^{\frac{2\pi i}{3}}$ so the eigenvectors (up to a scalar) of $\tau$ in $U_1$ are \begin{align*}
    v_1&=\delta_e+\delta_{(123)}+\delta_{(132)},\\v_2&=\omega\delta_e+\omega^2\delta_{(123)}+\delta_{(132)},\\v_3&=\omega^2\delta_e+\omega\delta_{(123)}+\delta_{(132)}.
    \end{align*}
    $\sigma\cdot\delta_{e}=\delta_{(12)},\sigma\cdot\delta_{(123)}=\delta_{(23)},\sigma\cdot\delta_{(132)}=\delta_{(13)}$ so
\begin{align*}
    \sigma\cdot v_1&=\delta_{(12)}+\delta_{(23)}+\delta_{(13)},\\\sigma\cdot v_2&=\omega\delta_{(12)}+\omega^2\delta_{(23)}+\delta_{(13)},\\\sigma\cdot v_3&=\omega^2\delta_{(12)}+\omega\delta_{(23)}+\delta_{(13)}.
    \end{align*}
\item[(c)] Note that $\sigma\cdot v_1$ is an eigenvector of $\tau$ with eigenvalue $1$, $\sigma\cdot v_2$ is an eigenvector of $\tau$ with eigenvalue $\omega$ and $\sigma\cdot v_3$ is an eigenvector of $\tau$ with eigenvalue $\omega^2$.

Let $V_1=\langle v_1+\sigma\cdot v_1\rangle$. Then $\tau\cdot(v_1+\sigma\cdot v_1)=v_1+\sigma\cdot v_1$ and $\sigma\cdot(v_1+\sigma\cdot v_1)=v_1+\sigma\cdot v_1$ so $V_1$ has dimension $1$ and both $\tau$ and $\sigma$ act as the identity.

Let $V_2=\langle v_1-\sigma\cdot v_1\rangle$. Then $\tau\cdot(v_1-\sigma\cdot v_1)=v_1-\sigma\cdot v_1$ and $\sigma\cdot(v_1-\sigma\cdot v_1)=-(v_1-\sigma\cdot v_1)$ so $V_1$ has dimension $1$, $\tau$ acts as the identity and $\sigma$ acts as $-\text{Id}_{V_2}$.

Finally, let $V_3=\text{span}\{v_2,\sigma\cdot v_2\}$ and let $V_4=\text{span}\{v_3,\sigma\cdot v_3\}$.

Since $(v_1,\sigma\cdot v_1),(v_2,\sigma\cdot v_3),(v_3,\sigma\cdot v_2)$ are linearly independent pairs of eigenvectors with distinct eigenvalues, we have that $\mathbb{C}S_3=\text{span}\{v_1,\sigma\cdot v_1\}\oplus\text{span}\{v_2,\sigma\cdot v_3\}\oplus\text{span}\{v_3,\sigma\cdot v_2\}$ as vector spaces. Thus $\mathbb{C}S_3=V_1\oplus V_2\oplus V_3\oplus V_4$.
\end{enumerate}
\end{proof}
\end{document}
