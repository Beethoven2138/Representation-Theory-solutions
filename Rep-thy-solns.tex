\documentclass{article}
\usepackage{graphicx} % Required for inserting images
\usepackage[english]{babel}
\usepackage{amssymb}
\usepackage{amsthm}
\usepackage{enumitem} 
\usepackage{amsmath}
\usepackage{amsfonts}
\usepackage{tikz}
\usetikzlibrary{matrix}
\usepackage[english]{babel}
\usepackage{mathtools}
\usepackage[a4paper, total={6in, 8in}]{geometry}
\usepackage{cite}
\title{Representation Theory Solutions}
\author{Saxon Supple}
\date{November 2024}
\usepackage{centernot}

\newtheorem{theorem}{Theorem}[section]
\newtheorem{definition}[theorem]{Definition}
\newtheorem{lemma}[theorem]{Lemma}
\newtheorem{proposition}[theorem]{Proposition}
\newtheorem{corollary}[theorem]{Corollary}
\newtheorem{example}[theorem]{Example}
\newtheorem{remark}[theorem]{Remark}
\newtheorem{exercise}[theorem]{Exercise}

\begin{document}

\maketitle

\begin{exercise}
Consider a basis of $V$ to be a linear isomorphism $\beta : \mathbb{F}^n \to V$ (where $n = \dim V$). Associating to a matrix $A \in \mathrm{GL}(n, \mathbb{F})$ the linear operator $\theta_A : \mathbb{F}^n \to \mathbb{F}^n$, show that the map
\[
\Phi : \mathrm{GL}(n, \mathbb{F}) \to \mathrm{GL}(V) : A \mapsto \beta \theta_A \beta^{-1}
\]
is an isomorphism of groups.
\end{exercise}
\begin{proof}
$\Phi$ is well-defined because the composition of isomorphisms is an isomorphism so $\beta\theta_A\beta^{-1}\in GL(V)\forall A\in GL(n,\mathbb{F})$. $\Phi(AB)=\beta \theta_{AB}\beta^{-1}=\beta \theta_{A}\beta^{-1}\beta \theta_{B}\beta^{-1}=\Phi(A)\Phi(B)$ so $\Phi$ is a homomorphism. Now let $\Phi(A)=\text{Id}_V$. Then $\beta \theta_A\beta^{-1}=\text{Id}_V\implies \theta_A\beta^{-1}=\beta^{-1}\text{Id}_V\implies \theta_A=\beta^{-1}\beta=\text{Id}_{\mathbb{F}^n}\implies A=I$. Thus $\Phi$ is injective. Given a $\phi\in GL(V)$ let $A$ be the matrix representing $\phi$ with respect to $\beta$. Then $\Phi(A)=\phi$. Thus $\Phi$ is surjective so is an isomorphism of groups.
\end{proof}

\begin{exercise}
Let $\phi: G \to \mathrm{GL}(V)$ be a representation of $G$. Show that the map
\[
\alpha: G \times V \to V : (g, v) \mapsto g \cdot v = \phi(g)(v)
\]
is a linear $G$-action.
\end{exercise}
\begin{proof}
Let $g,h\in G,v\in V$. Then $(gh)\cdot v=\phi(gh)(v)=\phi(g)(\phi(h)(v))=\phi(g)(h\cdot v)=g\cdot(h\cdot v)$.
$e\cdot v=\phi(e)(v)=\text{Id}_V(v)=v$. Thus $\alpha$ is a $G$-action.
Let $\lambda,\mu\in\mathbb{F}$ and $v,w\in V$. Then $g\cdot(\lambda v+\mu w)=\phi(g)(\lambda v+\mu w)=\lambda\phi(g)v+\mu\phi(g)w=\lambda(g\cdot v)+\mu(g\cdot w)$. Thus $\alpha$ is a linear $G$-action.
\end{proof}

\begin{exercise}
Let $\alpha : G \times X \to X : (g, x) \mapsto g \cdot x$ be a $G$-action and $\tilde{\alpha} : G \times \mathbb{F}X \to \mathbb{F}X : (g, f) \mapsto g \cdot f$ be its linearisation. Let $\{\delta_x : x \in X\}$ be the standard basis of $\mathbb{F}X$. 

\noindent Show that $g \cdot \delta_x = \delta_{g \cdot x}$. Deduce that the linearised action can be characterised by
\[
g \cdot \sum_{x \in X} \lambda_x \delta_x = \sum_{x \in X} \lambda_x \delta_{g \cdot x}.
\]
\end{exercise}
\begin{proof}
$(g\cdot\delta_x)(y)=\delta_x(g^{-1}\cdot y)$ which is $1$ iff $g^{-1}\cdot y=x\iff y=g\cdot x$ and $0$ otherwise. Thus $g \cdot \delta_x = \delta_{g \cdot x}$. By linearity, $g \cdot \sum_{x \in X} \lambda_x \delta_x=\sum_{x \in X} \lambda_x g\cdot\delta_x=\sum_{x \in X} \lambda_x \delta_{g \cdot x}$
\end{proof}


\begin{exercise}
Find $n$ different degree $1$ representations of $\mathbb{Z}_n$ over $\mathbb{C}$ and determine which are faithful.
\end{exercise}
\begin{proof}
$\rho_k:\mathbb{Z}_n\to\mathbb{C}^{*}:a\mapsto\omega^{ak}$ where $\omega=e^{\frac{2\pi i}{n}}$ and $0\leq k\leq n-1$ are $n$ degree $1$ representations. They're homomorphisms since $\rho_k(a+b)=\omega^{(a+b)k}=\omega^{ak}\omega^{bk}=\rho_k(a)\rho_k(b)$. $\rho_k$ is faithful when its image is the set of all $n$'th roots of unity which occurs when $\omega^k$ has order $n$ which occurs when $k$ is coprime to $n$.
\end{proof}

\begin{exercise}
Show that the only finite subgroups of the group $\mathbb{C}^*$ are the cyclic groups of $n$th roots of unity, for each positive integer $n$.
\end{exercise}
\begin{proof}
Let $H$ be a subgroup of $\mathbb{C}^*$ of order $n$ and let $h\in H$. Then $h^n=1$ by Lagrange's theorem so $h$ is an $n$th root of unity. Furthermore, since $H$ has order $n$ and only comprises $n$th roots of unity, $H$ must then contain all $n$th roots of unity and so be cyclic, generated by $e^{\frac{2\pi i}{n}}$.
\end{proof}

\begin{exercise}
The dihedral group $D_{2n}$ may be characterised as a group of order $2n$ generated by two elements $a$ and $b$, subject to the relations $a^n=b^2=1$ and $ba=a^{-1}b$.
\begin{enumerate}[label=(\roman*)]
    \item Show that any element of $D_{2n}$ can be written uniquely in the form $a^ib^j$ where $0\leq i<n$ and $0\leq j<2$.
    \item Now let $H$ be some other group with $h,k\in H$. What are necessary and sufficient conditions for there to exist a homomorphism $\theta:D_{2n}\to H$ with $\theta(a)=h$ and $\theta(b)=k$? Show that, then, such a homomorphism is unique.
    \item Show that $D_{2n}$ has a representation $\rho:D_{2n}\to GL(1,\mathbb{R})$ with $\rho(a)=1$ and $\rho(b)=-1$. What is the geometric significance of this representation?
    \item Write down a faithful matrix representation of $D_8$ of degree $2$.
\end{enumerate}
\end{exercise}
\begin{proof}
\begin{enumerate}[label=(\roman*)]
    \item Let $w$ be an element of $D_{2n}$ written as $x_1^{y_1}...x_n^{y_m}$ for $x_i\in\{a,b\},y_i\in\mathbb{Z}$ where $y_i=1$ if $x_i=b$. Suppose that there is some $x_i=b$ with $i\neq m$. Then $x_{i+1}=a$ so we can swap $b$ $y_{i+1}$ times with the rightwards $a$ so that $w=x_1^{y_1}...x_{i+1}^{-y_{i+1}}bx_{i+2}^{y_{i+2}}...x_{m}^{y_m}$. Again simplify if possible if $x_{i+2}=b$. Repeating this process will then give $w=a^{x}b^j$ where $0\leq j<2$. Then let $i=x+ln$ where $l\in\mathbb{Z}$ is such that $i$ is in the desired range. This shows that any element of $D_{2n}$ can be written in the desired form. To show uniqueness, note that $D_{2n}$ has order $2n$, each of the form $a^ib^j$ with $i$ and $j$ in the given ranges, and there are at most $2n$ possible elements given by $a^ib^j$ so we must have uniqueness in order to cover all $2n$ elements. 
    \item We must have $h^n=b^2=1$ and $kh=h^{-1}k$ as a necessary condition. $\theta$ is then given as $\theta(a^ib^j)=h^ik^j$. To show that $\theta$ is a homomorphism, let $v=a^i$ and $w=a^xb^y$. Then $\theta(vw)=\theta(a^ia^xb^y)=\theta(a^{i+x}b^y)=h^{i+x}k^y=h^ih^xk^y=\theta(v)\theta(w)$. Now let $v=a^ib$. Then $\theta(vw)=\theta(a^iba^xb^y)=\theta(a^{i-x}b^{y+1})=h^{i-x}k^{y+1}=h^ikh^xk^y=\theta(v)\theta(w)$. Thus the necessary conditions are also sufficient conditions for there to be a homomorphism. The homomorphism is also completely determined by its values in $a$ and $b$ so is unique.
    \item $a^n=(-1)^2=1$ and $(-1)1=1^{-1}(-1)$ so there exists such a representation. The representation encodes whether or not a reflection occurred on a regular $n$-gon.
    \item let $\rho(a)=\begin{pmatrix}
  \text{cos}(\frac{2\pi}{8}) & -\text{sin}(\frac{2\pi}{8})\\ 
  \text{sin}(\frac{2\pi}{8}) & \text{cos}(\frac{2\pi}{8})
\end{pmatrix}$ and let $\rho(b)=\begin{pmatrix}
  1 & 0\\ 
  0 & -1
\end{pmatrix}.$

$\rho(a^i)=\begin{pmatrix}
  \text{cos}(\frac{2\pi i}{8}) & -\text{sin}(\frac{2\pi i}{8})\\ 
  \text{sin}(\frac{2\pi i}{8}) & \text{cos}(\frac{2\pi i}{8})
\end{pmatrix}=I\iff i\in 8\mathbb{Z}\iff a^i=1$.

$\rho(a^ib)=\begin{pmatrix}
  \text{cos}(\frac{2\pi i}{8}) & \text{sin}(\frac{2\pi i}{8})\\ 
  \text{sin}(\frac{2\pi i}{8}) & -\text{cos}(\frac{2\pi i}{8})
\end{pmatrix}$ which is never $I$.

This $\rho$ has trivial kernel so is faithful.
\end{enumerate}
\end{proof}

\begin{exercise}
The quaternion group $Q_8$ is the subgroup of $GL(2,\mathbb{C})$ generated by the matrices\[A=\begin{pmatrix}
  i & 0\\ 
  0 & -i
\end{pmatrix},   B=\begin{pmatrix}
  0 & 1\\ 
  -1 & 0
\end{pmatrix}.\]
\begin{enumerate}[label=(\roman*)]
    \item Show that $A^4=1$, $A^2=B^2$ and $BAB^{-1}=A^{-1}$, and conclude that $Q_8$ is non-abelian.
    \item Show that any element of $Q_8$ can be written uniquely in the form $A^iB^j$ with $0\leq i<4$ and $0\leq j<2$. Thus confirm that $Q_8$ has order $8$.
    \item Is $Q_8$ isomorphic to $D_8$?
\end{enumerate}
\end{exercise}
\begin{proof}
\begin{enumerate}[label=(\roman*)]
    \item $BA=A^{-1}B\neq AB$.
    \item Let $W=X_1^{Y_1}...X_n^{Y_n}$ where $X_i$ is $A$ or $B$. Suppose that there is an $i$ such that $X_i=B$ and $X_{i+1}=A$. Then $W=X_1^{Y_1}...X_{i+1}^{-Y_{i+1}}X_i^{Y_i}...X_n^{Y_n}$. Repeat this until $W=A^pB^q$. Then let $r=p+4k$ such that $0\leq r<4$ and let $s=q+4k$ such that $0\leq s<4$. Then $W=A^rB^s$. If $s=0$ or $s=1$, let $i=r$ and $j=s$. If $s=2$, let $i=r+2\text{ mod }4$ and $j=0$. And if $s=3$, let $i=r+2\text{ mod }4$ and let $j=1$. Then $W=A^iB^j$ with $i$ and $j$ in the desired ranges.

    For uniqueness, it can be shown that $A$ and $B$ can generate $8$ distinct values, and each can be written as one of the $8$ possibilities of $A^iB^j$ so there can be no repetition so we have uniqueness.
    \item No. the elements of $Q_8$ has orders $1,4,2,4,4,4,4,4$ whereas the elements of $D_8$ have orders $1,4,2,4,2,$etc. The groups then have different numbers of elements of order $2$ so can't be isomorphic.
\end{enumerate}
\end{proof}

\begin{exercise}
Show that two matrix representations of degree one, $\rho_1:G\to GL(1,\mathbb{F})$ and $\rho_2:G\to GL(1,\mathbb{F})$ are isomorphic if and only if $\rho_1=\rho_2$.
\end{exercise}
\begin{proof}
$(\impliedby)$ Trivial. Take $\text{Id}_{\mathbb{F}}$ as the $G$-linear map.

$(\implies)$ Let $\theta:\mathbb{F}\to\mathbb{F}:v\mapsto\lambda v$ for $\lambda\in\mathbb{F}^*$ be a $G$-linear isomorphism. Then $\lambda\rho_1(g)=\theta\rho_1(g)=\rho_2(g)\theta=\lambda\rho_2(g)\implies\rho_1(g)=\rho_2(g)\forall g\in G$ so $\rho_1=\rho_2$.
\end{proof}

\begin{exercise}
Consider the real matrix representation of \( G = \mathbb{Z}_3 \) given by
\[
\rho : \mathbb{Z}_3 \to \mathrm{GL}(2, \mathbb{R}) : k \mapsto
\begin{pmatrix}
\cos \frac{2\pi k}{3} & -\sin \frac{2\pi k}{3} \\
\sin \frac{2\pi k}{3} & \cos \frac{2\pi k}{3}
\end{pmatrix}.
\]

Show that the corresponding \( G \)-module \( V = \mathbb{R}^2 \) is irreducible.

On the other hand, if we use the same matrices to define a complex representation
\[
\rho_{\mathbb{C}} : \mathbb{Z}_3 \to \mathrm{GL}(2, \mathbb{C}),
\]
show that the corresponding \( G \)-module \( V_{\mathbb{C}} = \mathbb{C}^2 \) is not irreducible.

\end{exercise}

\begin{proof}
Let $W\neq \{0\}$ be a $G$-submodule of $V$. Let $0\neq x\in W$. Then $\rho(1)(x)$ rotates $x$ by $\frac{2\pi}{3}$ radians and so $x$ and $\rho(1)(x)$ are linearly independent. Thus $\text{dim}(W)=2$ so $W=V$. Thus $V$ is irreducible.

Let $W=\{(\lambda,\lambda i),\lambda\in\mathbb{C}\}$. Given $\lambda\in\mathbb{C}^*$, $\rho(1)(\lambda,\lambda i)=\lambda(\text{cos}(\frac{2\pi}{3})-i\text{sin}(\frac{2\pi}{3}),\text{sin}(\frac{2\pi}{3})+i\text{cos}(\frac{2\pi}{3}))=\lambda(\text{cos}(\frac{2\pi}{3})-i\text{sin}(\frac{2\pi}{3}),(\text{cos}(\frac{2\pi}{3})-i\text{sin}(\frac{2\pi}{3}))i)=(\text{cos}(\frac{2\pi}{3})-i\text{sin}(\frac{2\pi}{3}))(\lambda,\lambda i)\in W$. Similarly, $\rho(2)(\lambda,\lambda i)\in W$ so $W$ is a $G$-submodule. $\text{dim}(W)=1<\text{dim}(\mathbb{C}^2)=2$ so $W\neq V_\mathbb{C}$. Thus $V_\mathbb{C}$ is not irreducible.
\end{proof}

\begin{exercise}
If \( G \) acts on a non-empty set \( X \), then the linearisation \( V = \mathbb{F}X \) contains two \( G \)-submodules:
\[
W_0 = \{ f \in \mathbb{F}X : \sum_{x \in X} f(x) = 0 \}, \quad
W_1 = \{ f \in \mathbb{F}X : f \text{ is constant} \}.
\]

Show that, if \( \mathrm{char} \, \mathbb{F} \) does not divide \( |X| \), then \( V \) is a direct sum \( V = W_0 \oplus W_1 \). What happens if \( \mathrm{char} \, \mathbb{F} \) does divide \( |X| \)?

\end{exercise}

\begin{proof}
Let $f\in W_0\cap W_1$. Then $f(x)=c\forall x\in X$ and $c|X|=0$. $|X|\neq0$ so $c=0$. Thus $W_0\cap W_1=\{0\}$. Let $f\in V$. Let $c=\sum_{x\in X}f(x)$. If $c=0$, then $f\in W_0$. Otherwise, define $g\in W_0$ by $g(x)=f(x)-\frac{c}{|X|}$ and $h\in W_1$ by $h(x)=\frac{c}{|X|}$. Then $f=g+h$. Thus $V=W_0\oplus W_1$.

If $\text{char}\mathbb{F}$ divides $|X|$ then $W_1\subseteq W_0$.
\end{proof}

\begin{exercise}
Let \( V \) and \( W \) be \( G \)-modules over a field \( \mathbb{F} \) and recall that 
\(\mathrm{Hom}_{\mathbb{F}}(V, W)\) denotes the vector space of all linear maps from \( V \) to \( W \).

\begin{itemize}
    \item[(a)] Show that \(\mathrm{Hom}_{\mathbb{F}}(V, W)\) becomes a \( G \)-module if we define \( g \cdot T \) by
    \[
    (g \cdot T)(v) = g \cdot \big(T(g^{-1} \cdot v)\big),
    \]
    for \( T \in \mathrm{Hom}_{\mathbb{F}}(V, W) \), \( g \in G \), and \( v \in V \).

    \item[(b)] The dual of \( V \) is \( V^* = \mathrm{Hom}_{\mathbb{F}}(V, \mathbb{F}) \). Show that \( V^* \) becomes a \( G \)-module if we define \( g \cdot f \) by
    \[
    (g \cdot f)(v) = f(g^{-1} \cdot v),
    \]
    for \( f \in V^* \), \( g \in G \), and \( v \in V \).
\end{itemize}

\end{exercise}

\end{document}
