\documentclass{article}
\usepackage{graphicx} % Required for inserting images
\usepackage[english]{babel}
\usepackage{amssymb}
\usepackage{amsthm}
\usepackage{enumitem} 
\usepackage{amsmath}
\usepackage{amsfonts}
\usepackage{tikz}
\usetikzlibrary{matrix}
\usepackage[english]{babel}
\usepackage{mathtools}
\usepackage[a4paper, total={6in, 8in}]{geometry}
\usepackage{cite}
\title{Representation Theory Solutions}
\author{Saxon Supple}
\date{November 2024}
\usepackage{centernot}

\newtheorem{theorem}{Theorem}[section]
\newtheorem{definition}[theorem]{Definition}
\newtheorem{lemma}[theorem]{Lemma}
\newtheorem{proposition}[theorem]{Proposition}
\newtheorem{corollary}[theorem]{Corollary}
\newtheorem{example}[theorem]{Example}
\newtheorem{remark}[theorem]{Remark}
\newtheorem{exercise}[theorem]{Exercise}

\begin{document}

\maketitle

\begin{exercise}
Consider a basis of $V$ to be a linear isomorphism $\beta : \mathbb{F}^n \to V$ (where $n = \dim V$). Associating to a matrix $A \in \mathrm{GL}(n, \mathbb{F})$ the linear operator $\theta_A : \mathbb{F}^n \to \mathbb{F}^n$, show that the map
\[
\Phi : \mathrm{GL}(n, \mathbb{F}) \to \mathrm{GL}(V) : A \mapsto \beta \theta_A \beta^{-1}
\]
is an isomorphism of groups.
\end{exercise}
\begin{proof}
$\Phi$ is well-defined because the composition of isomorphisms is an isomorphism so $\beta\theta_A\beta^{-1}\in GL(V)\forall A\in GL(n,\mathbb{F})$. $\Phi(AB)=\beta \theta_{AB}\beta^{-1}=\beta \theta_{A}\beta^{-1}\beta \theta_{B}\beta^{-1}=\Phi(A)\Phi(B)$ so $\Phi$ is a homomorphism. Now let $\Phi(A)=\text{Id}_V$. Then $\beta \theta_A\beta^{-1}=\text{Id}_V\implies \theta_A\beta^{-1}=\beta^{-1}\text{Id}_V\implies \theta_A=\beta^{-1}\beta=\text{Id}_{\mathbb{F}^n}\implies A=I$. Thus $\Phi$ is injective. Given a $\phi\in GL(V)$ let $A$ be the matrix representing $\phi$ with respect to $\beta$. Then $\Phi(A)=\phi$. Thus $\Phi$ is surjective so is an isomorphism of groups.
\end{proof}

\begin{exercise}
Let $\phi: G \to \mathrm{GL}(V)$ be a representation of $G$. Show that the map
\[
\alpha: G \times V \to V : (g, v) \mapsto g \cdot v = \phi(g)(v)
\]
is a linear $G$-action.
\end{exercise}
\begin{proof}
Let $g,h\in G,v\in V$. Then $(gh)\cdot v=\phi(gh)(v)=\phi(g)(\phi(h)(v))=\phi(g)(h\cdot v)=g\cdot(h\cdot v)$.
$e\cdot v=\phi(e)(v)=\text{Id}_V(v)=v$. Thus $\alpha$ is a $G$-action.
Let $\lambda,\mu\in\mathbb{F}$ and $v,w\in V$. Then $g\cdot(\lambda v+\mu w)=\phi(g)(\lambda v+\mu w)=\lambda\phi(g)v+\mu\phi(g)w=\lambda(g\cdot v)+\mu(g\cdot w)$. Thus $\alpha$ is a linear $G$-action.
\end{proof}

\begin{exercise}
Let $\alpha : G \times X \to X : (g, x) \mapsto g \cdot x$ be a $G$-action and $\tilde{\alpha} : G \times \mathbb{F}X \to \mathbb{F}X : (g, f) \mapsto g \cdot f$ be its linearisation. Let $\{\delta_x : x \in X\}$ be the standard basis of $\mathbb{F}X$. 

\noindent Show that $g \cdot \delta_x = \delta_{g \cdot x}$. Deduce that the linearised action can be characterised by
\[
g \cdot \sum_{x \in X} \lambda_x \delta_x = \sum_{x \in X} \lambda_x \delta_{g \cdot x}.
\]
\end{exercise}
\begin{proof}
$(g\cdot\delta_x)(y)=\delta_x(g^{-1}\cdot y)$ which is $1$ iff $g^{-1}\cdot y=x\iff y=g\cdot x$ and $0$ otherwise. Thus $g \cdot \delta_x = \delta_{g \cdot x}$. By linearity, $g \cdot \sum_{x \in X} \lambda_x \delta_x=\sum_{x \in X} \lambda_x g\cdot\delta_x=\sum_{x \in X} \lambda_x \delta_{g \cdot x}$
\end{proof}


\begin{exercise}
Find $n$ different degree $1$ representations of $\mathbb{Z}_n$ over $\mathbb{C}$ and determine which are faithful.
\end{exercise}
\begin{proof}
$\rho_k:\mathbb{Z}_n\to\mathbb{C}^{*}:a\mapsto\omega^{ak}$ where $\omega=e^{\frac{2\pi i}{n}}$ and $0\leq k\leq n-1$ are $n$ degree $1$ representations. They're homomorphisms since $\rho_k(a+b)=\omega^{(a+b)k}=\omega^{ak}\omega^{bk}=\rho_k(a)\rho_k(b)$. $\rho_k$ is faithful when its image is the set of all $n$'th roots of unity which occurs when $\omega^k$ has order $n$ which occurs when $k$ is coprime to $n$.
\end{proof}

\begin{exercise}
Show that the only finite subgroups of the group $\mathbb{C}^*$ are the cyclic groups of $n$th roots of unity, for each positive integer $n$.
\end{exercise}
\begin{proof}
Let $H$ be a subgroup of $\mathbb{C}^*$ of order $n$ and let $h\in H$. Then $h^n=1$ by Lagrange's theorem so $h$ is an $n$th root of unity. Furthermore, since $H$ has order $n$ and only comprises $n$th roots of unity, $H$ must then contain all $n$th roots of unity and so be cyclic, generated by $e^{\frac{2\pi i}{n}}$.
\end{proof}

\end{document}
